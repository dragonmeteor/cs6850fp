\documentclass[10pt, a4paper]{article}

\usepackage{fullpage}
\usepackage[english]{babel}
\usepackage[utf8x]{inputenc}
\usepackage{amsmath}
\usepackage{graphicx}
\usepackage[colorinlistoftodos]{todonotes}
\usepackage{url}
\usepackage{array}

\newcommand{\etal}{{et~al.}}

\title{Network Analysis of User-Generated Multimedia Contents:\\
A Case Study of MikuMikuDance Contents in Niconico}
\author{Derek Chou, Pramook Khungurn, Detian Shi, Chris Yu}

\begin{document}
\maketitle

\begin{abstract}
To be written later.
\end{abstract}

\section{Introduction}

We are interested in analyzing the network of videos and pictures in the Japanese content sharing site Niconico \cite{niconico}. 
We focus on contents created using a free 3D animation software MikuMikuDance \cite{mmd} (abbreviated as MMD). 
To date, there are at least 140,000 videos and 60,000 pictures uploaded to Niconico that are generated using or related to MMD. 

These items from a graph structure among themselves as they form directed links among themselves in two ways. 
First, the description of each item can contain links to other items just like web links. 
Second, Niconico contains a sub-website called Content Tree \cite{content-tree} that allows an item’s creator to specify which other items are the “parents” of her item. 
A link mostly points to the sources of materials used to create its origin and can be regarded as a citation, but links to other related items, such as the next item in a series, are also present. 

As with many types of user generated content, each piece of content is a conglomeration of other previous works. In particular, a video can make use of existing 3D models, motions, and tools, and very rarely does a user post a completed MMD video that is composed of parts that are wholly original. Rather, users usually specialize in a certain element of production, and the majority of videos are pieced together. The user activities giving rise to these artefacts are an example of large-scale collaborative content creation similar to academic research, and analyzing the graph of one subset of items is akin to analyzing the citation graph of an academic discipline.

\section{Previous Works}

Hamasaki \etal~\cite{Hamasaki:2008} analyzed a similar data set obtained from the same website, but focused on videos related to voice synthesizer application Hatsune Miku (which MikuMikuDance is named after). 
The application allows users to compose songs and have it performed digitally. 
It drew the interest of many aspiring artists and soon, a large community grew for both creation and consumption of the genre. 
The resulting videos thus not only include music and vocal tracks but also complex illustrations. 
A video that finally goes viral usually makes use of earlier works made by many different unrelated users. 

In their analysis of Hatsune Miku videos, Hamasaki \etal~used user-generated tags to classify the many different activities that goes into the final video into four categories: songwriting, song creation, illustration, and editing. They then went on to classify creators by the type of work they contribute and analyzed how a particular activity lead to activity of other creators. For example, they found that an action by a illustrator usually led to activity in other illustrators while activity by songwriters and song creators lead to activity in many different types of activities. We adopt this strategy in our analysis. 

\cite{Cha:2009} \cite{Gruhl:2004} \cite{Leskovec:2006}

\section{Data Collection and Processing}
We used the Niconico's APIs as well as website crawling to obtain item data such uploader's user ID, description, upload time, view count, comment count, and tags. Items also have a ``Mylist'' count, which can be interpreted as the number of times the item has been ``favourited.'' 

We limit the scope of our data collection to MMD-related items, defined as follows: we say that a tag is MMD-related if the tag contains ``MMD'' or ``MikuMikuDance'' as a substring, and we say that an item is MMD-related if it contains at least one MMD-related tag. 

As stated earlier, directed edges between items are found in two ways. First, we can an item's description for IDs of Niconico videos and images, and, for each ID found, we then create an edge from the item to the item corresponding to the found ID. Second, we examine the item's Content Tree page and create an edge from the item to each of its parent item. For this project, we do not distinguish between these two types of links and only create \emph{one} directed edge if both description link and Content link are found.

At the beginning of October, we started crawling Niconico website for MMD-related items by breadth-first searching from three seed items picked manually. After this first crawling, we added all items with tags related to MMD\footnote{We examined all the tags of the videos we found during the first crawl and filter out non-MMD-related tags. We then used the Niconico Search API to identify all items having any one of these tags and add them as seeds.} and performed breadth-first search again to obtain all reachable items. The crawling finished on November 12.

As this project analyzes information propagation among items, we exclude all edges that are ``forward in time'' --- that is, pointing from an item to another item uploaded after it. The resulting graph after this cleanup contains 245,103 MMD-related items and 414,154 directed edges among them.

Following \cite{Leskovec:2006}, we also infer the graph of creators who uploaded the items from the item graph. A directed edge in the creator graph is weighted by the number of links from videos of the source creator to the videos of the target creator. The creator graph contains 18,399 users and 176,324 edges.\footnote{In creating the graph, however, we did not take into account 80 items whose uploader cannot be identified due to being uploaded by a ``broadcasting channel'' rather than a regular website user. These items have 414 edges incident to them.}

A creator uploaded $13.32$ items on average. However, the distribution of users who uploaded a specific number of items follows a power law as can be seen in Figure~\ref{user-vs-item-count}.
\begin{figure}
	\centering
	\includegraphics[width=3in]{item_count_vs_user_plot.png}
	\caption{Plot of number of items uploaded versus the number of creators that uploaded that number of items, in log-log scale. It can be seen that the distribution follows a power law.}
	\label{user-vs-item-count}
\end{figure}

We also perfom other cleanups after activity classification in order to simplify further analysis of the graph. These are discussed in details in the next section.

\section{Activity Categorization} \label{activity-classification}

Creating 3D computer graphics content is a complex process that requires many types of expertise including 3D modeling, motion authoring, and movie direction. As a result, a creator of MMD videos rarely has all the above expertise and has to rely on data such as 3D models and motions that are provided by others. Creators who provide usable data also make announcements of their data's being publicly available by creating MMD videos featuring the usage, and creators who use the data sometimes give credit to the former by ``citing'' the announcement videos either in her own video description or in the Content Tree. \medskip

{\bf Activity categories.} As pointed out by Hamasaki \etal, classifying creators by what type of activity they engage in and then examining the number of edges between types of creators can reveal how creators depend on activities of one another. To gain similar insight in our case, we classify activity of MMD creators into 5 categories:
\begin{itemize}
	\item \emph{CG production}, where a creator uses MMD to create music videos, short animations, and images,
	\item \emph{modeling}, where a user create 3D models of characters, static scenes, and props,
	\item \emph{choreography}, where a user creates poses for 3D characters or key-framed motion data,
	\item \emph{tool creation}, where a user creates software tools, visual effects, other types of data that can be used by the MMD software, or instructional videos, and
	\item \emph{summarizing}, where a user create videos that summarize, introduce, or compare other MMD videos. This activity typically does not concern with the MMD software and involves only skills in video editing. 
\end{itemize}

{\bf Categorizing items.} To categorize creators, we first categorize the items. For each of the categories except CG production, we manually identified tags that contains keywords related to that class of activity. For example, a video that announces a new 3D model usually contains the tag ``Has MMD Model Distribution,'' and a tutorial video usually contains the tag ``MMD Lecture.'' We say that an item belongs to one of the four category if it contains a tag in the list of tags identified with that category. If an item does not belong to any of the four, we say that it is in the CG production category.

By the above criteria, we have found that there are 225,916 CG production items, 13,377 modeling items, 2,582 choreography items, 2,306 tool items, and 1,365 summarizing items. While an item can belong to more than one category, we found only 435 such items with 11,653 edges incident to them. Statistics of items that belong to only a single category are given in Figure~\ref{item-statistics-by-category}. We can readily observe that, items in the modeling, choreography, and tool making categories receive many more in-links than those in the other two categories, reflecting the practice of creators citing sources of the data they use in their works.\medskip
\begin{figure*}[t]
	\centering
	\begin{tabular}{c|r|r|r|r|r}
		Category & Items & In-Links & In-Links per Item & Out-Links & Out-Links per Item \\
		\hline
		CG Production & 225,916 & 129,250 & 0.57 & 368,444 & 1.63\\
		Modeling & 12,994 & 147,825 & 11.38 & 20,873 & 1.61\\
		Choreography & 2,364 & 55,730 & 23.57 & 6,094 & 2.58\\
		Tool Making & 2,052 & 68,714 & 33.49 & 4,060 & 1.98\\
		Summarizing & 1,342 & 982 & 0.73 & 3,030 & 2.26
	\end{tabular}
	\caption{Statistics of items that belong to only one category. Links were counted ignoring those between items uploaded by the same creator.}
	\label{item-statistics-by-category}
\end{figure*}

{\bf Categorizing creators.} We also classified each creator into one category based on the types of videos she contributed. To do so, for each creator and for each category, we count the number of items that belongs to the category that the creator contributes. As almost all creators contributed items in the CG production category, we say that a creator is a CG producer if they contribute no videos in other categories. For each of the remaining creators, we classify her into the one of the four remaining category that she contributes the most numbers of items to. Ties are broken arbitrarily in the following arbitrary order: tool making, then modeling, then choreography, and then summarizing.

We found that there are 15,092 are CG producers ($82.03\%$), 1,962 modelers ($10.66\%$), 654 choreographers ($3.55\%$), 339 tool makers ($2.17\%$), and 292 summarizers ($1.59\%$). We surmise that this might be due to (1) low interest in summarizing videos, and (2) the fact that 3D modeling, choreography, and tool making requires more skills and efforts than just using MMD to create animations with previously created data. Other statistics for each category can be found in Figure~\ref{creator-statistics-by-category} and the number of links between different categories of creators can be found in Figure~\ref{user-links}.

\begin{figure}[t]
	\centering
	\newcolumntype{A}{>{\centering\arraybackslash} m{1cm}}
	\newcolumntype{B}{>{\raggedleft\arraybackslash} m{1.5cm}}
	\newcolumntype{C}{>{\centering\arraybackslash} m{2.5cm}}
	\newcolumntype{D}{>{\raggedleft\arraybackslash} m{1.25cm}}	
	\newcolumntype{E}{>{\centering\arraybackslash} m{1.5cm}}	
	\begin{tabular}{C|D|D|B|D|B|D|B}
		Category & 
		\multicolumn{1}{c|}{Creators} & 
		\multicolumn{1}{c|}{Items} & 
		\multicolumn{1}{E|}{Items per Creator} & 
		\multicolumn{1}{c|}{In-Links} & 
		\multicolumn{1}{E|}{In-Links per Item} & 
		\multicolumn{1}{c|}{Out-Links} & 
		\multicolumn{1}{E}{Out-Links per Item} \\
		\hline
		CG Producers & 15,092 & 124,408 & 8.24 & 14,703 & 0.12 & 184,966 & 1.49\\
		Modelers & 1,962 & 76,303 & 38.89 & 193,135 & 2.53 & 103,093 & 1.35\\
		Choreographers & 654 & 26,157 & 40.00 & 71,136 & 2.72 & 45,196 & 1.73\\
		Tool Makers & 399 & 13,823 & 34.64 & 79,090 & 5.72 & 19,712 & 1.43\\
		Summarizers & 292 & 4,332 & 14.84 & 2,036 & 0.47 & 7,133 & 1.65
	\end{tabular}
	\caption{Creator statistics by categories. 53,647 self-citations were ignored when counting links.}
	\label{creator-statistics-by-category}
\end{figure}

\begin{figure}[t]
\centering
\begin{tabular}{c|r|r|r|r|r}
 & CG producers & Modelers & Choreographers & Tool Makers & Summarizers \\
 \hline
CG Producers & 7,683 & 101,470 & 40,171 & 34,818 & 824 \\
Modelers & 3,885 & 53,658 & 18,876 & 26,092 & 582 \\
Choreographers & 1,526 & 24,506 & 8,377 & 10,491 & 296 \\
Tool Makers & 645 & 9,826 & 2,524 & 6,545 & 172 \\
Summarizers & 964 & 3,675 & 1,188 & 1,144 & 162 \\
\end{tabular}
\caption{Number of item links between users of different categories, not counting self-citations. The datum in each cell indicates the number of links from the \emph{row} cateogory to the \emph{column} category.}
\label{user-links}
\end{figure}

One interesting observation is that the modelers, choreographers, and tool makers upload many more items compared to other types of creators. Due to the fact that there are few videos in the three corresponding categories, we can infer that these three types of creators are actually very productive CG producers who can actually make their own resources and chose to share the resources to the community. Moreover, the large number of in-links they acquire are likey not through the CG productions they created but rather through the items containing reusable data. Thus, despite their small numbers, the modelers, choreographers, and tool makers are the life-blood of the community for two reasons. First, they account for almost a half of all the items produced, and, second, they provide reusable data for others to create more works.

\section{Network Topology}

{\bf Number of components.} The item graph contains 96,744 components. The user graph contains 5,249 components. \medskip

\section{Item Popularity}
Our data set contains four indicators of item popularity: view count, comment count, Mylist count, and the indegree. \medskip

{\bf Distributions.} As can be seen in Figure~\ref{distribution-of-popularity}, all popularity indicators except the view count follows power law distributions. The view count distribution exhibits a peak at $226$ views, suggesting that, even unpopular items usually accumulates several hundreds views. This phenomenon can be observed in the Flickr network studied by Cha \etal\ as well.\footnote{Cha \etal, however, plotted the cumulative distribution of item count versus view count. In their graph, the cumulative distribution plateaus when view count is low, which suggests that there are few items with low view counts too.} \medskip
\begin{figure}[t]
	\centering
	\begin{tabular}{cccc}
		\includegraphics[width=1.5in]{view_count_vs_item_count_plot.png} &
		\includegraphics[width=1.5in]{comment_count_vs_item_count_plot.png} &
		\includegraphics[width=1.5in]{mylist_count_vs_item_count_plot.png} &
		\includegraphics[width=1.5in]{indegree_vs_item_count_plot.png}\\
		view count & comment count & Mylist count & indegree
	\end{tabular}
	\caption{Distribution of item popularity.}
	\label{distribution-of-popularity}
\end{figure}

{\bf Correlation between popularity indicators.} The view count, the comment count, and the Mylist count all have strong positive correlations among one another.\footnote{When calculating the correlations, we excluded one video which has disproportionately large number of comments (approximately 10,800,000 while the item with second most number of comments only has about 120,000 comments). This video was made specifically to fish a large number of comments. It urges the viewer to leave a comment by clicking a built-in button in the video.} However, the indegree has low correlations to all other indicators. (See Figure~\ref{correlation-table}.) We surmise that this is due to the fact that because the indegree measures popularity among creators, who tend to link to items that contain reusable data, not general viewers. \medskip
\begin{figure}
	\centering
	\begin{tabular}{c|c|c|c|c}
		& V & C & M & I \\
		\hline
		V & 1 & 0.82 & 0.93 & 0.16 \\
		\hline
		C & 0.82 & 1 & 0.76 & 0.09 \\
		\hline
		M & 0.93 & 0.76 & 1 & 0.14 \\
		\hline
		I & 0.16 & 0.09 & 0.14 & 1 \\
	\end{tabular}
	\caption{Correlation between four indicators of item popularity. Here, V = view count, C = comment count, M = Mylist count, and I = indegree.}
	\label{correlation-table}
\end{figure}

{\bf Most in-linked items.} We examined the 30 most in-linked items and found 17 items in the tool making category, 6 in the modeling category, and 7 in the CG category. However, upon closer inspection, all of the items in the CG category contains web links to either character models or dance motions. As such, these items should have been classified as being in the modeling or choreogrphy category, but we could not classify them as such because they lacked relevant tags. 

The most in-linked item is the video introducing MMD itself, and the 9 most in-linked items all provide some kind of tools or data for creating visual effects such as luminous objects, shadow, or the sky. The total number of in-links to the 17 most popular tool videos is 42,379 which is about $62\%$ of all the in-links items in this category have. While this phenomenon can be attributed to a strong rich-get-richer effect, we suggest that these tools that are fundamental to CG production and thus are used by many creators in the community. \medskip

{\bf Temporal evolution of indegree.} Inspecting the 30 most in-linked items, we observed three distinct patterns of indegree evolution over time as depicted in Figure~\ref{temporal-evolution-of-indegree-01}. First is the ``convex'' evolution where the rate the item gains more links increases as the item gets older. Second is the ``linear'' case where the item always steadly gains in-links at rates that do not fluctuate wildly. Third is the ``concave'' case where the item rapidly gains many in-links in the first few days after it was uploaded and then the gain rate decreases with time.
\begin{figure}
	\centering
	\begin{tabular}{cccc}
		\includegraphics[width=2in]{indegree_evolution/0001.png} &
		\includegraphics[width=2in]{indegree_evolution/0015.png} &
		\includegraphics[width=2in]{indegree_evolution/0017.png} \\
		Convex &
		Linear &
		Concave		
	\end{tabular}
 	\caption{The three patterns of indegree growth.}
 	\label{temporal-evolution-of-indegree-01}
\end{figure}

Among the 30 items, we observe that the tool making items often exhibit convex growth pattern. We conjecture that this is because these items are useful in general to all creators and that the community of MMD creators gets larger with time. On the other hand, the concave growth pattern is exclusive to modeling and choreography video, suggesting that, after a new popular 3D model or motion become available, it instantly becomes a fad, inducing many creators to try their hands on it. However, it time goes on, the interest subsides due to the availability of other models and motions.

In contrast to Cha \etal, we do not observe any stagnations among the 30 most popular videos, and we only observe sharp increase in indegree only after an item was uploaded. This suggests that a reusable piece of data will always be constantly be used and cited. Moreover, unlike favourites in Flickr, the indegree evolution in our graph is not influenced by external events such as a photo's being featured or winning an award.

We also attempted to classify the indegree growth patterns of items by computing the sum $$A(T) = \sum_{t=0}^T \mathrm{indeg}(t) - \frac{t}{T} \mathrm{indeg}(T)$$ where $T$ is a fixed number of days after the item was uploaded. We classify the growth pattern as:
\begin{itemize}
 	\item ``linear'' if $|A(T)| \leq (T+1) \mathrm{indeg}(T) / 20$, 
 	\item ``concave'' if $A(T) > (T+1)\mathrm{indeg}(T) / 20$, and
 	\item ``convex'' if $A(T) < -(T+1)\mathrm{index}(T)/20$. 
\end{itemize}
The intuition is that the sum $A(T)$ is the area under the curve of the difference between the indegree and the perfect linear growth over the period of $T+1$ days. If the indegree has linear growth, this area should be close to zero. If the indegree has concave growth, it should mostly be above the graph of the perfect linear growth and the area should be highly positive. Similarly, convex growth implies highly negative area under the curve. 

The results of classifying the growth patterns of ``popular'' items---those with indegree more than 20---over the period of 1 year after they are uploaded are shown in Figure~\ref{growth-patterns-of-popular-items}. We can see that the ratio of items with convex growth patterns in the tool making category is the highest among all categories, and that the ratio of items with concave growth patterns are much higher in all other categories. This confirms our prior observation among the 30 most popular items that tool making items tend to exhibit linear or convex growth, but items in other categories tend to exhibit concave growth.
\begin{figure}
	\centering
	\begin{tabular}{c|r|r|r|r}
	& \multicolumn{3}{c|}{Growth Pattern} & \\
	\cline{2-4}
	Category & \multicolumn{1}{c|}{Convex} & \multicolumn{1}{c|}{Linear} & \multicolumn{1}{c|}{Concave} & \multicolumn{1}{c}{Total} \\
	\hline
	CG Production & 37 & 58 & 223 & 318 \\
	Modeling & 73 & 153 & 327 & 553 \\
	Choreography & 10 & 50 & 225 & 285 \\
	Tool Making & 39 & 39 & 41 & 119 \\
	Summarizing & 0 & 0 & 0 & 0 \\
	\hline
	All & 159 & 300 & 816 & 1,275
	\end{tabular}	
	\caption{Number of items that achieved more than 20 in-links after 1 year they were uploaded, classified by growth patterns and categories.}
	\label{growth-patterns-of-popular-items}
\end{figure}

We also investigated the indegree evolution of ``unpopular items,'' i.e. those achieved at most 20 in-links after 1 year after uploaded. Figure~\ref{unpopular-items-achieving-cerntain-fraction-of-integree} shows the number of items achieving a certain percentage of their indegree at the end of the period of 1 year as a function of time. According to Figure~\ref{unpopular-items-achieving-cerntain-fraction-of-integree}a, more than half of the items obtained all their in-links within the first month after being uploaded. However, we found that more than half of the unpopular items ($17,658$ out of $30,025$) have self-citations as their only in-links. Discounting self-citations, we found that it takes longer for items to gain all their in-links from others, as can be seen in Figure~\ref{unpopular-items-achieving-cerntain-fraction-of-integree}b. Still, growth pattern classification (results not shown in this paper) indicates that most items exhibit concave growth patterns.
\begin{figure}
	\centering
	\begin{tabular}{cc}
		\includegraphics[width=2.5in]{unpopular_items_indegree_growth_with_self_link.png} &
		\includegraphics[width=2.5in]{unpopular_items_indegree_growth_no_self_link.png} \\
		(a) & (b)
	\end{tabular}
	\caption{Number of items achieving a certain percentage of their indegree over the period of one year as a function of elapsed time after being uploaded. In (a), the indegree calculation includes self-citations but not in (b).}
	\label{unpopular-items-achieving-cerntain-fraction-of-integree}
\end{figure}

In conclusion, most items gains indegree early in their lifetime and thus exhibit concave patterns of indegree growth. However, popular items in the tool making category are more likely to gain in-links at a faster rate as time progresses. Most unpopular items gain in-links via self-citations, which typically happen within the first month after the item was uploaded. This is most likely due to creators' linking to their previous work when uploading a new one. Discounting self-citations, it takes much longer time for an unpopular item to gain a link from other creators.
 
\section{Cascade Patterns}

As defined in \cite{Gruhl:2004}, a cascade is a maximal directed acyclic subgraph rooted at a node with no outgoing edges, the \emph{cascade originator}. A cascade represents how information from the cascade originator spreads to other items. Leskovec \etal~observe that cascades has two main shapes: the \emph{star} and the \emph{chain}.

In our graph, a link from one item to another is likely an indication that the linking item, in its creation, uses some piece of data from the linked item. As such, we may predict that cascade originators in our graph are likely to be modeling, choreography, and tool making items. Since data from these items can be used to create many different works, we expect that a cascade originator should generally have many items linking it. We also observed in Section~\ref{activity-classification} that there are many more CG production items which rarely provide reusable data than those that do. As a result, an item linking to a cascade originator is likely to be a CG production item and likely has no in-links. That is, we should see more stars than chains in general.

To put this hypothesis to a test, we implemented the algorithm to count cascade pattern as described in \cite{Leskovec:2006}. However, when performing graph isomorphism tests, we consider two nodes the same if and only if they have the same categories. In effect, each node in our cascade pattern can have one of the five \emph{colors} determined by its category, and two cascades are considered the same if and only if their topologies and colors match.

The distribution of the cascade patterns of the full graph is given in Figure~\ref{cascade-patterns-with-self-links}.
\begin{figure}
	\centering
	\begin{tabular}{ccccccc}
		\includegraphics[scale=0.5]{cascades_with_self_links/0001.png} &
		\includegraphics[scale=0.5]{cascades_with_self_links/0002.png} &
		\includegraphics[scale=0.5]{cascades_with_self_links/0003.png} &
		\includegraphics[scale=0.5]{cascades_with_self_links/0004.png} &
		\includegraphics[scale=0.5]{cascades_with_self_links/0005.png} &
		\includegraphics[scale=0.5]{cascades_with_self_links/0006.png} &
		\includegraphics[scale=0.5]{cascades_with_self_links/0007.png} \\		
		$G_1$: 5839 &
		$G_2$: 795 &
		$G_3$: 607 &
		$G_4$: 589 &
		$G_5$: 307 &
		$G_6$: 272 &
		$G_7$: 169 \\		
		\includegraphics[scale=0.5]{cascades_with_self_links/0008.png} &
		\includegraphics[scale=0.5]{cascades_with_self_links/0009.png} &		
		\includegraphics[scale=0.5]{cascades_with_self_links/0010.png} &
		\includegraphics[scale=0.5]{cascades_with_self_links/0011.png} &
		\includegraphics[scale=0.5]{cascades_with_self_links/0012.png} &
		\includegraphics[scale=0.5]{cascades_with_self_links/0013.png} &
		\includegraphics[scale=0.5]{cascades_with_self_links/0014.png} \\		
		$G_8$: 129 &
		$G_9$: 102 &
		$G_{10}$: 93 &
		$G_{11}$: 88 &
		$G_{12}$: 85 &
		$G_{13}$: 72 &
		$G_{14}$: 71 \\
		\includegraphics[scale=0.5]{cascades_with_self_links/0015.png} &
		\includegraphics[scale=0.5]{cascades_with_self_links/0016.png} &
		\includegraphics[scale=0.5]{cascades_with_self_links/0017.png} &		
		\includegraphics[scale=0.5]{cascades_with_self_links/0018.png} &
		\includegraphics[scale=0.5]{cascades_with_self_links/0019.png} &
		\includegraphics[scale=0.5]{cascades_with_self_links/0020.png} &
		& \\
		%\includegraphics[scale=0.5]{cascades_with_self_links/0021.png} \\
		$G_{15}$: 70 &
		$G_{16}$: 68 &
		$G_{17}$: 65 &
		$G_{18}$: 57 &
		$G_{19}$: 56 &
		$G_{20}$: 55 &
		%$G_{21}$: 63 
	\end{tabular}
	\caption{}
	\label{cascade-patterns-with-self-links}
\end{figure}

\begin{figure}
	\centering
	\begin{tabular}{ccccc}
		\includegraphics[scale=0.5]{cascades_no_self_links/0001.png} &
		\includegraphics[scale=0.5]{cascades_no_self_links/0002.png} &
		\includegraphics[scale=0.5]{cascades_no_self_links/0003.png} &
		\includegraphics[scale=0.5]{cascades_no_self_links/0004.png} &
		\includegraphics[scale=0.5]{cascades_no_self_links/0005.png} \\
		$G_1$: 2700 &
		$G_2$: 887 &
		$G_3$: 595 &
		$G_4$: 455 &
		$G_5$: 435 &		
		\includegraphics[scale=0.5]{cascades_no_self_links/0006.png} &
		\includegraphics[scale=0.5]{cascades_no_self_links/0007.png} &
		\includegraphics[scale=0.5]{cascades_no_self_links/0008.png} &
		\includegraphics[scale=0.5]{cascades_no_self_links/0009.png} &		
		\includegraphics[scale=0.5]{cascades_no_self_links/0010.png} \\		
		$G_6$: 191 &
		$G_7$: 184 &
		$G_8$: 138 &
		$G_9$: 134 &
		$G_{10}$: 132 &		
		\includegraphics[scale=0.5]{cascades_no_self_links/0011.png} &
		\includegraphics[scale=0.5]{cascades_no_self_links/0012.png} &
		\includegraphics[scale=0.5]{cascades_no_self_links/0013.png} &
		\includegraphics[scale=0.5]{cascades_no_self_links/0014.png} &
		\includegraphics[scale=0.5]{cascades_no_self_links/0015.png} \\		
		$G_{11}$: 121 &
		$G_{12}$: 85 &
		$G_{13}$: 84 &
		$G_{14}$: 83 &
		$G_{15}$: 74 \\		
		\includegraphics[scale=0.5]{cascades_no_self_links/0016.png} &
		\includegraphics[scale=0.5]{cascades_no_self_links/0017.png} &		
		\includegraphics[scale=0.5]{cascades_no_self_links/0018.png} &
		\includegraphics[scale=0.5]{cascades_no_self_links/0019.png} &
		\includegraphics[scale=0.5]{cascades_no_self_links/0020.png} \\
		$G_{16}$: 67 &
		$G_{17}$: 64 &
		$G_{18}$: 60 &
		$G_{19}$: 54 &
		$G_{20}$: 52 		
	\end{tabular}
	\caption{}
	\label{cascade-patterns-no-self-links}
\end{figure}

\bibliographystyle{plain}
\bibliography{main}  

\end{document}